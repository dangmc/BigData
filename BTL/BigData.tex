% Copyright 2004 by Till Tantau <tantau@users.sourceforge.net>.
%
% In principle, this file can be redistributed and/or modified under
% the terms of the GNU Public License, version 2.
%
% However, this file is supposed to be a template to be modified
% for your own needs. For this reason, if you use this file as a
% template and not specifically distribute it as part of a another
% package/program, I grant the extra permission to freely copy and
% modify this file as you see fit and even to delete this copyright
% notice. 

\documentclass{beamer}

% There are many different themes available for Beamer. A comprehensive
% list with examples is given here:
% http://deic.uab.es/~iblanes/beamer_gallery/index_by_theme.html
% You can uncomment the themes below if you would like to use a different
% one:

\usetheme{Madrid}
\usepackage{multirow}
\usepackage{cases} 
\usepackage{graphicx} %package to manage images
\graphicspath{ {/home/dangmc/Documents/Learning/20161/BigData/BTL/Report/Image/} }
\usepackage[utf8]{inputenc}
\usepackage[vietnam]{babel}
\usepackage{enumitem}
\usepackage{algorithm,algorithmic}
\title[Recommendation System]{Xây dựng hệ thống dự đoán sự đánh giá một bộ phim của người xem}

% A subtitle is optional and this may be deleted
\subtitle{}

\author[DangMc, P.H.Hạnh]{\qquad  Sinh viên thực hiện \\ \qquad \vspace{1cm}Đặng Mạnh Cường \ Phan Thị Hồng Hạnh  \\
\qquad Giáo viên hướng dẫn \\ 
\qquad  TS.Trần Vĩnh Đức}

\institute[HUST] % (optional, but mostly needed)
{
 % from CNTT2-04\\
  %Ha Noi University of Science and Technology
  }
% - Give the names in the same order as the appear in the paper.
% - Use the \inst{?} command only if the authors have different
%   affiliation.

\date{\qquad Hà Nội, 09-1-2017}

\AtBeginSubsection[]
{
  \begin{frame}<beamer>{Mục Lục}
    \tableofcontents[currentsection,currentsubsection]
  \end{frame}
}
\newenvironment<>{varblock}[2][\textwidth]{%
  \setlength{\textwidth}{#1}
  \begin{actionenv}#3%
    \def\insertblocktitle{#2}%
    \par%
    \usebeamertemplate{block begin}}
  {\par%
    \usebeamertemplate{block end}%
  \end{actionenv}}
% Let's get started
\begin{document}
\begin{frame}
  \titlepage
\end{frame}

\begin{frame}{Mục Lục}
  \tableofcontents
  % You might wish to add the option [pausesections]
\end{frame}

% Section and subsections will appear in the presentation overview
% and table of contents.
\section{Nội dung đề tài}
\begin{frame}{Nội dung đề tài}
\begin{itemize}[label = \textbullet]
	\item Xây dựng hệ thống có khả năng dự đoán sự đánh giá (rating) của người xem (user) đối với một bộ phim (movie)
		
	\item Dựa vào lịch sử đánh giá của người xem đối với các bộ
 phim - tập huấn luyện
 	\begin{itemize}[label = \textendash]
			\item Ma trận rating $R_{mxn}$, m là số lượng người xem, n là số lượng bộ phim
			\item Nếu $R_{xi} \ne null$, người xem x đánh giá phim i với giá trị $R_{xi}$ sao ($R_{xi} \in [0, 5]$)
			\item Nếu $R_{xi} = null$, người xem x chưa đánh giá phim i   
		\end{itemize}		
\end{itemize}
\end{frame}
\begin{frame}{Nội dung đề tài}
	Đánh giá hiệu quả của hệ thống:
		\begin{itemize}[label = \textbullet]
			\item Sử dụng tập Test: Tập T gồm các cặp người dùng, bộ phim mà hệ thống cần dự đoán
			\begin{center}
				$RMSE = \sqrt{\frac{\sum_{(i, x) \in T}{(T_{xi} - \hat{T_{xi}})^2}}{|T|}} \rightarrow$ $MIN$ \\

				\begin{itemize}[label = \textendash]
					\item $\hat{T_{xi}}$ là đánh giá của người xem x đối với bộ phim i do hệ thống dự đoán 
					\item $T_{xi}$ là đánh giá thực của người xem x đối với bộ phim i 
				\end{itemize}
			\end{center}
		\end{itemize}
\end{frame}
\section{Mô hình}
\subsection{Lọc cộng tác}
\begin{frame}{Ý tưởng}{Lọc cộng tác}
	\begin{itemize}[label = \textbullet]
		\item $U$ là tập người xem
		\item $M$ là tập các bộ phim
		\item Xét người dùng x và bộ phim i
		\item Trong tập $M(x)$ gồm những bộ phim mà người xem x đã rating, tìm tập $H(x,i)$ gồm những bộ phim có rating tương đồng nhất với i. 
		\item Ước lượng rating của người xem x đối với bộ phim i dựa vào tập $H(x,i)$
	\end{itemize}

\end{frame}
\begin{frame}{Công thức đo độ tương đồng}{Lọc cộng tác}
	\begin{itemize}[label = \textbullet]
		\item Gọi $sim(i,j)$ là độ tương đồng giữa i và j 
		\item Sử dụng độ đo $cosine$:
		$sim(i, j) = cos(R_i, R_j) = \frac{R_iR_j}{||R_i||||R_j||}$ 
		
	\begin{figure}
    		\centering
		\includegraphics[width=0.7\textwidth]{cosine}
		\label{fig:img1}
	\end{figure}
	\hspace{2cm}
	$sim(A, B) = \frac{4x5}{\sqrt{4^2+5^2+1^2}\sqrt{5^2+5^2+4^2}} = 0.38$
	\item Khi $sim(i,j)$ càng lớn thì độ tương đồng giữa i và j càng cao  
	\end{itemize}
\end{frame}
\begin{frame}{Tìm tập tương đồng}{Lọc cộng tác}
	\begin{itemize}[label = \textbullet]
		\item Với mỗi bộ phim $i$, tìm tập $N(i) = \{j \in M | sim(i, j) \ge 0.2 \}$ \\ \hspace{2cm}
		$\Rightarrow H(x,i) = N(i) \cap M(x) $
		\item Làm sao để tìm tập N(i)?
			\begin{itemize}[label = \textendash]
				\item sử dụng thuật toán tầm thường $\Rightarrow$ độ phức tạp quá lớn
				\item sử dụng Minhashing + Locality Sensitive Hashing
			\end{itemize}
	\end{itemize}
\end{frame}
\begin{frame}{Tìm tập tương đồng}
	\begin{figure}
    		\centering
		\includegraphics[width=0.7\textwidth]{lsh}
		\label{fig:img1}
	\end{figure}
\end{frame}
\begin{frame}{Minhashing}{LSH}
	\begin{figure}
    		\centering
		\includegraphics[width=0.5\textwidth]{mh}
		\label{fig:img1}
	\end{figure}
	\begin{itemize}[label = \textbullet]
		\item $R_i$ là vector rating của bộ phim i ($|R_i| = m$)
		\item Lấy 1 tập ngẫu nhiên các vector $\{v_1, v_2,...,v_k\}$ kích thước m, các vector chỉ chứa 2 giá trị -1 hoặc 1
		\item Xây dựng vector chữ ký $S_i$ cho bộ phim i như sau:
		\[
 \forall j \in [1, k], S_{ji} = 
  \begin{cases} 
   1 & \text{if } R_iv_j > 0 \\
   0       & \text{if } R_iv_j \le 0
  \end{cases}
\]
	\end{itemize}
	
\end{frame}
\begin{frame}{LSH}
	\begin{figure}
    		\centering
		\includegraphics[width=0.5\textwidth]{b}
		\label{fig:img1}
	\end{figure}
	\begin{itemize}[label = \textbullet]
		\item Chia ma trận $S$ thành $b$ băng, mỗi băng $r$ hàng
		\item Với mỗi băng, tiến hành băm phần của cột chữ ký thuộc băng đó vào bảng băm gồm c giỏ
		\item Một cặp ứng cử viên là một cặp được băm vào cùng 1 giỏ trong 1 băng bất kì
		\item Với mỗi cặp ứng cử viên, tính độ tương đồng và xây dựng tập $N(i)$
	\end{itemize}
\end{frame}
\begin{frame}{Ước lượng rating}

\begin{center}
	$\hat{R_{xi}} = b_{xi} + \frac{\sum_{j \in H(x,i)}{sim(i,j)x(R_{xj} - b_{xj})}}{\sum_{j \in H(x,i)}{sim(i,j)}}$
\end{center}
Trong đó: \\
\begin{itemize}[label = \textbullet]
	\item $b_{xi} = \mu + b_x + b_i$
	\item $\mu$ là giá trị trung bình rating trên toàn ma trận R
	\item $b_x$ là chênh lệch giữa giá trị rating trung bình của người xem x với $\mu$
	\item $b_i$ là chênh lệch giữa giá trị rating trung bình của bộ phim i với $\mu$
\end{itemize}

\end{frame}
\subsection{Nhân tố ẩn}
\begin{frame}{Nhân tố ẩn}
	\begin{itemize}[label = \textbullet]
		\item Gọi k là số nhân tố ẩn
		\item $P_{kxm}$ là ma trận đặc tính tiềm ẩn của người xem
		\item $Q_{kxn}$ là ma trận đặc tính tiềm ẩn của các bộ phim
		\item Cần tìm $P,Q$ sao cho $P^{T}Q \approx R$
	\end{itemize}
\end{frame}
\begin{frame}{Nhân tố ẩn}
	Cực tiểu hàm mục tiêu: \\ 
		$E = \sum\limits_{(x,i) \in R}{(R_{xi} - \hat{R_{xi}})^2 + \frac{\lambda}{2}[\sum\limits_{x}{||P||^2} + \sum\limits_{i}{||Q||^2} + \sum\limits_{x}{||U||^2} + \sum\limits_{i}{||I||^2}]}$
	Trong đó: \\
	\begin{itemize}[label = \textbullet]
		\item $\hat{R_{xi}} = \mu + U_{x} + I_{i} + P_x^TQ_{i}$
		\begin{itemize}[label = \textendash]
			\item $\mu$ là giá trị trung bình rating của ma trận R	
			\item $U_{x}$ là giá trị bias của người xem x
			\item $I_{i}$ là giá trị bias của bộ phim i
		\end{itemize}
		\item $\lambda$ là tham số điều khiển 
	\end{itemize}
\end{frame}
\begin{frame}{Stochastic gradient descent}
	Đặt $W = \frac{\lambda}{2}[\sum\limits_{x}{||P||^2} + \sum\limits_{i}{||Q||^2} + \sum\limits_{x}{||U||^2} + \sum\limits_{i}{||I||^2}] $ \\
	Xét người xem x và bộ phim i: \\
	$E_{xi} = (R_{xi} - \hat{R_{xi}})^2 + W
	 = [R_{xi} - (\mu + U_x + I_i + \sum\limits_{t = 1}^k{P_{tx}Q_{ti}})]^2 + W$ \\
	$\varepsilon_{xi} = 2 * (R_{xi} - \mu - U_x - I_i - \sum\limits_{t = 1}^k{P_{tx}Q_{ti}})$ \\
	Cập nhật tham số:
	\begin{itemize}[label = \textbullet]
		\item $P_{tx} = P_{tx} - \alpha\frac{\partial E_{xi}}{\partial P{tx}} = P_{tx} - \alpha(-\varepsilon_{xi}Q_{ti} + \lambda P_{tx})$
		\item $Q_{ti} = Q_{ti} - \alpha\frac{\partial E_{xi}}{\partial Q{ti}} = Q_{ti} - \alpha(-\varepsilon_{xi}P_{tx} + \lambda Q_{ti})$
		\item $U_x = U_x - \alpha\frac{\partial E_{xi}}{\partial U_x} = U_x - \alpha(-\varepsilon_{xi} + \lambda U_x)$
		\item $I_i = I_i - \alpha\frac{\partial E_{xi}}{\partial I_i} = I_i - \alpha(-\varepsilon_{xi} + \lambda I_i)$
	\end{itemize}

\end{frame}
\begin{frame}{Stochastic gradient descent}{Nhân tố ẩn}
	\begin{algorithm}[H]
		\begin{algorithmic}[1]
			\STATE $Initialize$ $P, Q, U, I$
			\FOR{$(x, i) \in training$}
				\STATE $\varepsilon_{xi} = 2 * (r_{xi} - \mu - U_x - I_i - P_x^TQ_i)$
				\STATE $P_x = P_x + \alpha*(\varepsilon_{xi}Q_i - \lambda P_x)$
				\STATE $Q_i = Q_i + \alpha*(\varepsilon_{xi}P_x - \lambda Q_i)$
				\STATE $U_x = U_x + \alpha*(\varepsilon_{xi} - \lambda U_x)$
				\STATE $I_i = I_i + \alpha*(\varepsilon_{xi} - \lambda I_i)$
			\ENDFOR
		\end{algorithmic}
		\caption{Stochastic gradient descent}
		\label{alg:seq}
	\end{algorithm}
\end{frame}
\section{Kết quả thực nghiệm}
\begin{frame}{Bộ dữ liệu}
	\begin{itemize}[label = \textbullet]
		\item MovieLen - 100k: bộ dữ liệu gồm 100000 rating của 942 người xem trên 1692 bộ phim
		\item MovieLen - latest: bộ dữ liệu gồm hơn 24 triệu rating của 256000 người xem trên 40110 bộ phim
	\end{itemize}
\end{frame}
\begin{frame}{Kết quả}
Bảng kết quả so sánh giá trị hàm RMSE giữa 2 mô hình lọc cộng tác (CF) và nhân tố ẩn (LF).
\begin{center}
\begin{tabular}{|c|c|c|c|c|}
\hline
 &\multicolumn{2}{c|}{ml-100k} & \multicolumn{2}{c|}{ml-latest}  \\
\cline{2-5}
 & Traing data & Test data & Training data & Test data  \\
\hline
CF & 0.85  & 0.95 & 0.98 & 1.03 \\
\hline
LF & 0.41 & 0.92& 0.90 & 0.92  \\
\hline
% etc. ...
\end{tabular}
\end{center}
$\Rightarrow$ Mô hình nhân tố ẩn cho kết quả tốt hơn lọc cộng tác trên cả 2 bộ dữ liệu
\end{frame}
\begin{frame}{Kết quả}
	Đồ thị biểu diễn quá trình học LF trên bộ dữ liệu ml-100k
	\begin{figure}
    		\centering
		\includegraphics[width=0.9\textwidth]{ml-100k}
		\label{fig:img1}
	\end{figure}
\end{frame}
\begin{frame}{Kết quả}
	Đồ thị biểu diễn quá trình học LF trên bộ dữ liệu ml-latest
	\begin{figure}
    		\centering
		\includegraphics[width=0.9\textwidth]{ml-latest}
		\label{fig:img1}
	\end{figure}
\end{frame}
\begin{frame}
	\Huge
	\begin{center}
		THANK YOU!
	\end{center}
\end{frame}
\end{document}


